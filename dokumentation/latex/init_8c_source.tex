\hypertarget{init_8c_source}{}\section{init.\+c}
\label{init_8c_source}\index{styr/styr/init.\+c@{styr/styr/init.\+c}}

\begin{DoxyCode}
00001 \textcolor{comment}{/*}
00002 \textcolor{comment}{ *        File: init.c}
00003 \textcolor{comment}{ *     Version: 1.0}
00004 \textcolor{comment}{ * Last edited: 15/4/2016 }
00005 \textcolor{comment}{ *     Authors: erilj291 }
00006 \textcolor{comment}{ */} 
00007 
00008 
00009 \textcolor{preprocessor}{#include "\hyperlink{init_8h}{init.h}"}
00010 
\hypertarget{init_8c_source.tex_l00011}{}\hyperlink{init_8h_a7ce0a14b6e7779fbb2d9a05333792c41}{00011} \textcolor{keywordtype}{void} \hyperlink{init_8c_a7ce0a14b6e7779fbb2d9a05333792c41}{Init}(\textcolor{keywordtype}{void})
00012 \{
00013     DDRD = 1<<DDD2;
00014     DDRC = 0; \textcolor{comment}{//JTAG, alla v�ljs till ing�ngar}
00015     DDRB = (1<<DDB3) | (1<<DDB4) | (1<<DDB5) | (0<<DDB6) | (1<<DDB7); \textcolor{comment}{//SPI, allt ut f�rutom PB6}
00016     
00017     PORTD |= 1<<PORTD2; \textcolor{comment}{//V�lj riktning "till servon" i tri-state}
00018     
00019     \hyperlink{init_8c_ad5cbed2a2222bb84e8b5c1caaa50634e}{UART\_Init}();
00020     \hyperlink{_s_p_i_8c_a49ddfef1b082be9cebd28e4bbebf247d}{SPI\_init\_master}();
00021 \}
00022 
\hypertarget{init_8c_source.tex_l00023}{}\hyperlink{init_8h_ad5cbed2a2222bb84e8b5c1caaa50634e}{00023} \textcolor{keywordtype}{void} \hyperlink{init_8c_ad5cbed2a2222bb84e8b5c1caaa50634e}{UART\_Init}(\textcolor{keywordtype}{void})
00024 \{
00025     \textcolor{comment}{//Se till att baud-inst�llningarna blir r�tt}
00026     UCSR0A &= ~(1<<U2X0);
00027     \textcolor{comment}{//S�tt baud-prescaler = 0 s� att baud-rate blir 1 MBPS}
00028     UBRR0H = 0;
00029     UBRR0L = 0;
00030     \textcolor{comment}{//St�ll in processorn som s�ndare och mottagare p� bussen}
00031     UCSR0B = (1<<RXEN0)|(1<<TXEN0);
00032     \textcolor{comment}{//8 data, 1 stopbit}
00033     UCSR0C = (0<<USBS0)|(3<<UCSZ00);
00034 \}
\end{DoxyCode}
