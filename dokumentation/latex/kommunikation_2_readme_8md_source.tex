\hypertarget{kommunikation_2_readme_8md_source}{}\section{kommunikation/\+Readme.md}

\begin{DoxyCode}
00001 # Kommunikationsmodul
00002 
00003 Här finns kod som hör till kommunikationsmodulen.
\end{DoxyCode}
